\thesisdetails{
نام و نام‌خانوادگی=ندا ایزدیان,
شماره دانشجویی=۹۶۱۲۱۴۱۰۱۸,
عنوان=بررسی کلاس منیفلدهای لندزبرگی تعمیم‌یافته, %فرمت نگارش پایان‌نامه/رساله دانشگاه قم,
دانشکده=علوم پایه,
گروه=ریاضیات, 
رشته=ریاضی محض,
گرایش=هندسه,
استاد راهنمای اوّل=دکتر اکبر طیبی (دانشیار),
استاد راهنمای دوّم=دکتر حسن نجومی(دانشیار),
استاد مشاور اوّل=دکتر مرتضی میرزایی(استادیار),
استاد مشاور دوّم=دکتر علیرضا توکلی(استادیار),
نماینده تحصیلات تکمیلی=دکتر سیداحمد فقیهی  (دانشیار),
تاریخ اتمام=مهر ۱۳۹۶,
تاریخ دفاع=۱۳۹۶/۰۷/۱۴,
تعداد واحد=6, 
نمره=19.25, 
نمره به حروف=نوزده و بیست و پنج صدم, 
درجه=عالی, % گزینه‌های ممکن عالی، بسیار خوب، خوب، قابل قبول'
داور داخلی اوّل=دکتر نسرین صادق‌زاده (استادیار),
داور داخلی دوّم=استاد داور داخلی دوّم(استاد),
داور خارجی اوّل=استاد داور خارجی اوّل(استاد),
داور خارجی دوّم=استاد داور خارجی دوّم(استاد),
author=Neda Izadian, 
title=On the class of generalized Landsberg Manifolds, % University of Qom's Thesis Style, 
faculty=Science,
department=Mathematics, 
major=Pure Mathematics, 
field=Geometry, 
submission date=November 2017, 
first supervisor=Dr. Akbar Tayebi,
%second supervisor=Dr. Hasan Nojumi,
first advisor=Dr. Morteza Mirzaie,
%second advisor=Dr. Alireza Tavakoli,
abstract=%----------------------------------------------------------------------------------------
%در این قسمت چکیده انگلیسی آورده می‌شود.
%----------------------------------------------------------------------------------------

In $2000$, Bejancu-Farran introduced the class of generalized Landsberg  manifolds which contains the class of Landsberg manifolds. In this thesis, we prove three global results for generalized Landsberg manifolds. First, we show that every compact generalized Landsberg manifold is a Landsberg manifold. Then we prove that every complete generalized landsberg manifold with relatively isotropic landsberg curvature reduces to a Landsberg manifold. Finally, we show that every generalized Landsberg manifold with vanishing Douglas curvature satisfies $ H=0 $. 
,
keywords={Landsberg Manifold, Riemannian Curvature, H-Curvature, Berwald Metric.  },%{Thesis, Style, XePersian, }, 
تقدیم به={ %بجای اینکه در این نقطه مطالب را ذکر کنید می‌توانید توضیحات را درون یک فایل نوشته و آن را در اینجا \input نمایید مانند شیوه‌ای که برای abstract در دو خط بالاتر بکار رفت. 
تقدیم به همسر و فرزندان عزیزم

که در این راه مرا تحمل نموده و صبورانه همراهی کردند 

\begin{traditionalpoem}
          تا ذوق درونم خبری می‌دهد از دوست &  از طعنه دشمن به خدا گر خبرستم \\
          می‌خواستمت پیشکشی لایق خدمت  &   جان نیک حقیرست ندانم چه فرستم \\
\end{traditionalpoem}
}, % پایان تقدیم به 
نیایش={\zarfont
منّت خدای را عز و جل که طاعتش موجب قربتست و به شکر اندرش مزید نعمت، هر نفسی که فرو می رود ممدّ حیاتست و چون بر می آید مفرّح ذات. 
پس در هر نفسی دو نعمت موجودست و بر هر نعمتی شکری واجب.
\begin{traditionalpoem}
از دست و زبان که برآید & کز عهده شکرش به در آید
\end{traditionalpoem}
اِعملوا آلَ داودَ شکراً وَ قلیلٌ مِن عبادیَ الشکور 

\begin{traditionalpoem}
بنده همان به که ز تقصیر خویش &  عذر به درگاه خدای آورد \\
ورنه سزاوار خداوندیش  &  کس نتواند که به جای آورد\\
\end{traditionalpoem}

باران رحمت بی حسابش همه را رسیده و خوان نعمت بی‌دریغش همه جا کشیده پرده ناموس بندگان به گناه فاحش ندرد و وظیفه روزی به خطای منکر نبرد

\begin{traditionalpoem}
ای کریمی که از خزانه غیب & گبر و ترسا وظیفه خور داری\\
دوستان را کجا کنی محروم & تو که با دشمن این نظر داری\\
\end{traditionalpoem}

}, % پایان نیایش
سپاسگزاری={\zarfont 
با تشکر از معاونت محترم آموزشی که موجبات فراهم آمدن چنین بسته‌ای را ممکن ساختند. 

اگر تلاش‌های شبانه‌روزی و بی‌شائبهٔ وفا خلیقی (توسعه دهنده بستهٔ فاخر زی‌پرشین) در طی ۱۲ سال اخیر نبود، امروز آماده‌سازی 
متون علمی پارسی در لاتک قطعاً با مشقات زیادی همراه بوده و شاید در نظر برخی تا حدی ناممکن می‌نمود. لذا قدردان زحمات بی‌منّت او 
بوده و برای او در هر کجای گیتی که باشد آرزوی سلامتی داریم. این استایل از ایده‌های دکتر خیلقی بهره‌های بسیار برده است. 

همچنین لازم است از کاربران گروه پارسی‌لاتک نیز تشکر به عمل آوریم که در طی سالیان اخیر با پاسخگویی به سوالات کاربران راهگشای ایشان بوده‌اند. 
},  % پایان سپاسگزاری
چکیده=%\input{abs},
{\zarfont  
چکیده شامل خلاصه‌ای از هدف یا مسأله پژوهش، روش شناسی، نتایج و تفسیر می‌‌شود که خواننده با مطالعه آن از محتوای
پژوهش آگاه می‌شود. در چکیده از اشاره به تاریخچه، تفصیل اقوال، توصیف تکنیک‌ها، فصل‌بندی، ذکر منابع و آوردن فرمول‌ها،
نمودارها و جداول پرهیز می‌شود. متن چکیده حداکثر باید 300 کلمه باشد و در یک صفحه و در یک بند (پاراگراف) نگاشته شود.
همچنین واژگان کلیدی در یک سطر جداگانه درج می شود و تعداد آن بین 5 تا 8 کلمه می‌باشد.},
کلمات کلیدی={چکیده،‌ پایان‌نامه، رساله، شیوه‌نامه، زی‌پرشین},
} % end of \thesisdetails macro
 
 
 
 \newcommand{\wi}[1]{\index{#1}#1}
\newcommand{\wil}[1]{\index{\lr{#1}}\lr{#1}}

% تنظیمات ذیل برای درج کد لاتک در سند استفاده شده است. 
\lstset{% general command to set parameter(s)
    float=htbp,
    language=[LaTeX]tex,
    numbers=left, 
    numberstyle=\tiny\yasfont,
    frame=single,
    frameround=fttt,
    gobble=0,
    breaklines=true,
%    escapechar={|},
    aboveskip=2\medskipamount,
}

